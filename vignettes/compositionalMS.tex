\documentclass[]{article}
\usepackage{lmodern}
\usepackage{amssymb,amsmath}
\usepackage{ifxetex,ifluatex}
\usepackage{fixltx2e} % provides \textsubscript
\ifnum 0\ifxetex 1\fi\ifluatex 1\fi=0 % if pdftex
  \usepackage[T1]{fontenc}
  \usepackage[utf8]{inputenc}
\else % if luatex or xelatex
  \ifxetex
    \usepackage{mathspec}
  \else
    \usepackage{fontspec}
  \fi
  \defaultfontfeatures{Ligatures=TeX,Scale=MatchLowercase}
\fi
% use upquote if available, for straight quotes in verbatim environments
\IfFileExists{upquote.sty}{\usepackage{upquote}}{}
% use microtype if available
\IfFileExists{microtype.sty}{%
\usepackage{microtype}
\UseMicrotypeSet[protrusion]{basicmath} % disable protrusion for tt fonts
}{}
\usepackage[margin=1in]{geometry}
\usepackage{hyperref}
\hypersetup{unicode=true,
            pdftitle={compositionalMS},
            pdfauthor={Jonathon O'Brien},
            pdfborder={0 0 0},
            breaklinks=true}
\urlstyle{same}  % don't use monospace font for urls
\usepackage{graphicx,grffile}
\makeatletter
\def\maxwidth{\ifdim\Gin@nat@width>\linewidth\linewidth\else\Gin@nat@width\fi}
\def\maxheight{\ifdim\Gin@nat@height>\textheight\textheight\else\Gin@nat@height\fi}
\makeatother
% Scale images if necessary, so that they will not overflow the page
% margins by default, and it is still possible to overwrite the defaults
% using explicit options in \includegraphics[width, height, ...]{}
\setkeys{Gin}{width=\maxwidth,height=\maxheight,keepaspectratio}
\IfFileExists{parskip.sty}{%
\usepackage{parskip}
}{% else
\setlength{\parindent}{0pt}
\setlength{\parskip}{6pt plus 2pt minus 1pt}
}
\setlength{\emergencystretch}{3em}  % prevent overfull lines
\providecommand{\tightlist}{%
  \setlength{\itemsep}{0pt}\setlength{\parskip}{0pt}}
\setcounter{secnumdepth}{0}
% Redefines (sub)paragraphs to behave more like sections
\ifx\paragraph\undefined\else
\let\oldparagraph\paragraph
\renewcommand{\paragraph}[1]{\oldparagraph{#1}\mbox{}}
\fi
\ifx\subparagraph\undefined\else
\let\oldsubparagraph\subparagraph
\renewcommand{\subparagraph}[1]{\oldsubparagraph{#1}\mbox{}}
\fi

%%% Use protect on footnotes to avoid problems with footnotes in titles
\let\rmarkdownfootnote\footnote%
\def\footnote{\protect\rmarkdownfootnote}

%%% Change title format to be more compact
\usepackage{titling}

% Create subtitle command for use in maketitle
\newcommand{\subtitle}[1]{
  \posttitle{
    \begin{center}\large#1\end{center}
    }
}

\setlength{\droptitle}{-2em}

  \title{compositionalMS}
    \pretitle{\vspace{\droptitle}\centering\huge}
  \posttitle{\par}
    \author{Jonathon O'Brien}
    \preauthor{\centering\large\emph}
  \postauthor{\par}
      \predate{\centering\large\emph}
  \postdate{\par}
    \date{2018-10-22}


\begin{document}
\maketitle

The compositionalMS package performs analyses on TMT/iTRAQ data under
the assumption that all of the relevant quantitative information is
captured in the signal proportions (O'Brien et al. 2018). The study of
constrained data is the primary focus of compositional data analysis
(Aitchison 1986) and the models used in this software are based on
prinicples from this field.

A \(D\) dimensional composition is defined as any vector \(\bf{x}\) of
non-negative elements \(x_{1}, ..., x_{D}\) such that
\(x_{1} + ... + x_{D} = K\).

The closure of a composition is the function that converts a composition
into a set of proportions \[
cl(\bf{x}) = \frac{x_{1}}{K}, ..., \frac{x_{D}}{K}
\]

\subsection{Normalization}\label{normalization}

\subsubsection{Row Normalization}\label{row-normalization}

The models used in this software anticipate singal-to-noise (SN) ratios,
which are theoretically greater than or equal to one. This is important
because the mathematical foundation of compositional data analysis does
not allow components to take a value of zero. While SN ratios should not
be less than one, various software packages will occasionally estimate
signals below the noise. Accordingly, the first step in our
normalization is to replace all SN values which are less than one with a
1.

Each scan is then reduced to its proportional information through the
additive log-ratio (ALR) transformation which maps a D dimensional
simplex into a vector in \(\mathbb{R}^{D-1}\).

\[
alr(\bf{x}) = \bf{y} = log_2(\frac{x_{2}}{x_{1}}), ..., log_2(\frac{x_{D}}{x_{1}})
\] Notice that the ALR is typically defined with the natural logarithm,
but we have defined it here as a log base 2 transformation to match the
conventional log-ratio scale used throughout the field of proteomics.

A transformation back into a the simplex (contrained to one), can be
achieved with the ALR inverse \[
alr^{-1}(\bf{y}) = cl(2^{0}, 2^{y_{1}}, ... 2^{y_{D-1}})
\]

The first component in the ALR transformation plays a special role as a
common reference for all of the log-ratios. Our program treats whatever
condition is provided in the first column as the reference. In
experiments that include a bridge channel it is natural to use the
bridge as the reference. When a bridge is not present, or in any case
where the first condition has more than one replicate, all columns
belonging to that first condition are collapsed into one column by
taking the geometric mean. Notice that converting each scan into
additive log-ratios with respect to a bridge channel provides a solution
to the cross-plex normalization problem.

\subsubsection{Column Normalization}\label{column-normalization}

In a mass spectrometry proteomics experiment it is common to assume that
the average protein abundance across samples should be equivalent.
Typically normalization is done by finding multiplicative factors such
that the mean (or sum) of each column is forced to be equivalent. This
should compensate for pipetting errors, variation in precipitation and
other factors that systematically shift the abundance of all proteins in
a sample.

If the function parameters include `normalize = TRUE' then a similar
normalization occurs that is slightly modified to use operations defined
in compositional data analysis. Specifically, each column is multiplied
so that the geometric means of the columns will be equivalent.

Column and row normalized matrices containing proportions are stored for
each plex. These are the values that are plotted when clicking on a
specific biological replicate. They should be interpreted as the
proportion of signal in the relevant plex that belonged to the peptide
shown.

\subsection{Model Specification}\label{model-specification}

Stastical modeling is performed on the normalized, ALR transformed data.
The model is defined as follows:

\[
y_{ijk} \sim N(\beta_{ij}, \sigma_{ij})\\
\beta_{ij} \sim N(0, 10) \\
\sigma_{ij} \sim InverseGamma(2, \tau)\\
\tau \sim halfNormal(0,5)\\
\] where \(y_{ijk}\) is the additive log-ratio, from the reference
condition, of the \(k\)'th petpide \((k = 1,..., n_{ij})\) nested within
the \(i\)'th protein \((i = 1,..., N)\) in the \(j\)'th condition
\((j = 1,...,M)\). \(\beta_{ij}\) represents the log-ratio (fold-change)
of the \(i\)'th protein from the reference to the \(j\)'th condition.
\(\sigma_{ij}\) is the prior standard deviation of the \(ij\)'th protein
fold-change and it is modeled to come from a distribution of errors
centered at \(\tau\). This yields a
\href{http://mc-stan.org/users/documentation/case-studies/pool-binary-trials.html}{partially
pooled variance estimate} for each protein fold-change, which provides
an effective solution for estimating the variance of proteins with only
a small number of observations.

The prior distribution \(\beta_{ij} \sim N(0, 10)\) was selected to be
weakly informative. The sampling is efficient since it does not have to
consider impossible outcomes, but the prior will have virtually zero
effect on the posterior estimate since a Gaussian distribution centered
at zero with a standard deviation of 10 covers a range of values greatly
exceeding the dynamic range of a TMT proteomics experiment (even
infinite changes usually appear to have log2 fold-changes less than 6 or
7 (O'Brien et al. 2018)). Similarly, \(\tau\), which represents the
average experimental error, has a half-normal prior distribution that
far exceeeds the full range of plausible experimental error while
preventing model instability caused by sampling unrealistic values. We
use an inverse-gamma distribution with shape parameter of 2 in order to
create a prior that enables stable sampling while covering all of the
realistic possibile errors for an experiment with values that tend fall
exclusively within (-10, 10). The shape parameter of 2 prevents sampling
of errors near zero which are unrealistic and can result in unstable
sampling, it also has the convienient property that the scale parameter,
\(\tau\) reprensents the mean of the distribution. Some of the
properties of this prior distribution have been previously explored
(Chung et al. 2013).

Partial pooling of the variance parameters results in estimates that
converge to the within group (protein/condition combination) variance
estimate as the number of observations increases. As the number of
observations converges to zero the estimate is pulled towards the
overall experimental error, \(\tau\). When there is only one observation
we have noticed some instability in these estimates. For this reason we
have programmed the model to directly use \(\tau\) as the standard
deviation for proteins with only a single peptide.

This Bayesian model is fit with the Stan programming languange
(Carpenter et al. 2017). The reported estimate and variance for each
protein are calculated as the mean and variance of the relevant
posterior distribution.

\subsection{Second Model}\label{second-model}

The above model provides estimates of log-fold changes for each protein
from a reference to each condition in the experiment. Each of these
estimates has variability from both within-sample peptide level variance
and between sample variation across replicates. In many experimental
designs the variation between biological replicates can large and
separating out this source of heterogeneity can be highly informative.
For this reason, by default, we fit two models. The above model which
estimates log2 fold-changes between conditions and another which
estimates log2 fold-changes between biological replicates. Estimates
from the condition model are shown in the Precicion Plots, while
Proportion Plots are generated by taking the \(ALR^{-1}\) of the
posterior distributions from the model that estimates changes for each
biological replicate.

It should be noted that when analyzing \(D\) dimensional compositions,
the covariance matrices are degenerate because there are only \(D-1\)
degrees of freedom. Consequently, variance estimates for the proportions
obtained on each of \(D\) biological replicates should be interpreted
with caution. For each of the \(D-1\) replicates that had a posterior
distributions in log-ratio space, the variance estimates come from a 1-1
transformation of the posterior distribution and should be reliable.
However, it is difficult to interpret the credible intervals on the new
variables (usually proteins from the bridge channel). We considered
dropping this interval from the plots, however we have found that
variation in the bridge across plexes can result in larger than usual
credible intervals. While these do not have the same interpretation as
the other replicates, they can still be informative.

\section*{References}\label{references}
\addcontentsline{toc}{section}{References}

\hypertarget{refs}{}
\hypertarget{ref-Aitchison1986}{}
Aitchison, John. 1986. \emph{The Statistical Analysis of Compositional
Data}. Chapman; Hall.

\hypertarget{ref-Carpenter2017}{}
Carpenter, Bob, Andrew Gelman, Matthew D. Hoffman, Daniel Lee, Ben
Goodrich, Michael Betancourt, Marcus Brubaker, Jiqiang Guo, Peter Li,
and Allen Riddell. 2017. ``Stan : A Probabilistic Programming
Language.'' \emph{Journal of Statistical Software} 76 (1): 1--32.
doi:\href{https://doi.org/10.18637/jss.v076.i01}{10.18637/jss.v076.i01}.

\hypertarget{ref-Chung2013}{}
Chung, Yeojin, Sophia Rabe-Hesketh, Vincent Dorie, Andrew Gelman, and
Jingchen Liu. 2013. ``A Nondegenerate Penalized Likelihood Estimator for
Variance Parameters in Multilevel Models.'' \emph{Psychometrika}.
doi:\href{https://doi.org/10.1007/s11336-013-9328-2}{10.1007/s11336-013-9328-2}.

\hypertarget{ref-OBrien2018}{}
O'Brien, Jonathon J., Jeremy D. O'Connell, Joao A. Paulo, Sanjukta
Thakurta, Christopher M. Rose, Michael P. Weekes, Edward L. Huttlin, and
Steven P. Gygi. 2018. ``Compositional Proteomics: Effects of Spatial
Constraints on Protein Quantification Utilizing Isobaric Tags.''
\emph{Journal of Proteome Research} 17 (1). American Chemical Society:
590--99.
doi:\href{https://doi.org/10.1021/acs.jproteome.7b00699}{10.1021/acs.jproteome.7b00699}.


\end{document}
